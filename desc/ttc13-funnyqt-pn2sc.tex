\documentclass[11pt]{article}
\usepackage[utf8]{inputenc}
\usepackage[T1]{fontenc}
\usepackage{hyperref}
\usepackage{palatino}
\usepackage{paralist}
\usepackage{verbatim}
\usepackage{footmisc}
\usepackage[a4paper,top=2cm,bottom=2cm,left=2.5cm,right=2.5cm]{geometry}

\makeatletter
\def\verbatim@font{\ttfamily\small}
\makeatother

\usepackage{minted}
\newminted{clojure}{fontsize=\footnotesize,frame=lines,linenos}


\title{Solving the Petri-Nets to Statecharts Transformation Case with FunnyQT}
\author{Tassilo Horn\\
  \href{mailto:horn@uni-koblenz.de}{horn@uni-koblenz.de}\\
  Institute for Software Technology\\
  University Koblenz-Landau}

\clubpenalty = 10000
\widowpenalty = 10000
\displaywidowpenalty = 10000

%% Reduce the space between image and captions
\setlength\abovecaptionskip{0.1cm}
\setlength\belowcaptionskip{0cm}


\begin{document}

\maketitle

\begin{abstract}
  This paper describes the FunnyQT solution to the TTC 2013 Petri-Nets to
  Statcharts Transformation Case.

  FunnyQT is a model querying and model transformation library for the
  functional Lisp-dialect Clojure.  It supports the modeling frameworks JGraLab
  and EMF natively, and it is designed to be extensible towards supporting
  other frameworks as well.

  FunnyQT provides a rich and efficient querying API, a model manipulation API,
  and on top of those, there are several sub-APIs for implementing several
  kinds of transformations such as ATL-like model transformations or programmed
  graph transformations.

  For solving this case, the model transformation API has been used for the
  initialization transformation, while the reduction transformation has been
  tackled algorithmically using the plain querying and model manipulation APIs.
\end{abstract}

\section{Introduction}
\label{sec:introduction}

\emph{FunnyQT} is a new model querying and transformation approach.  Instead of
inventing yet another language with its own concrete syntax and semantics, it
is implemented as an API for the functional, JVM-based Lisp-dialect
Clojure\footnote{\url{http://clojure.org/}}.  It's JVM-basing provides
wrapper-free access to all existing Clojure and Java libraries, and to other
tools in the rich Java ecosystem such as profilers.

FunnyQT natively supports the de-facto standard modeling framework EMF
\cite{Steinberg2008EEM} and the TGraph modeling framework
JGraLab\footnote{\url{http://jgralab.uni-koblenz.de}}, and it is designed to be
extensible towards other frameworks as well.

FunnyQT's API is split up in several sub-APIs.  On the lowest level there is a
core API for any supported modeling framework providing functions for loading
and storing models, accessing, creating, and deleting model elements, and
accessing and setting attribute values.  These core APIs mainly provide a
concise and expressive interface to the native Java APIs of the frameworks.  On
top of that, there's a generic API providing the subset of core functionality
that is common to both supported frameworks such as navigation via role names,
access to and manipulation of element properties, or functionality concerned
with typing imposed by metamodels.  Furthermore, there is a generic quering API
providing important querying concepts such as quantified expressions, regular
path expressions, or pattern matching.

Based on those querying and model manipulation APIs, there are several sub-APIs
for implementing different kinds of transformations.  For example, there is a
model transformation API similar to ATL \cite{ATL05} or ETL
\cite{booklet:epsilon}, or there is an in-place transformation API for writing
programmed graph transformations similar to GrGen.NET
\cite{manual:GrGenManual}.

Especially the pattern matching API and the transformation APIs make use of
Clojure's Lisp-inherited metaprogramming facilities
\cite{Graham1993OnLisp,Hoyte08LoL} in that they provide macros creating
internal DSLs \cite{book:Fowler2010DSL} providing concise, boilerplate-free
syntaxes to users.  Patterns and transformations written in these internal DSLs
get transformed to usual Clojure code using the FunnyQT querying and model
manipulation APIs by the Clojure compiler.

For solving the tasks of this transformation case\footnote{This FunnyQT
  solution is available at \url{https://github.com/tsdh/ttc-2013-pn2sc}
  and on SHARE (Section~\ref{sec:run-transformation})\label{fn:github}},
FunnyQT's model transformation API has been used for the initialization
transformation discussed in Section~\ref{sec:init-transformation}, while the
reduction transformation explained in
Section~\ref{sec:reduction-transformation} has been tackled algorithmically
using the plain querying and model manipulation APIs.


\section{The Initialization Transformation}
\label{sec:init-transformation}

The complete initialization transformation is shown in Listing~\ref{lst:init}.
It uses FunnyQT's model transformation API.

A transformation is declared with the \verb|deftransformation| macro.  It
receives the name of the transformation, i.e., \verb|initialize-statechart|, a
vector of input and output models, and arbitrary many rules.

The argument vector declares that the transformation receives one single input
model \verb|pn| which is an EMF model, and it receives exactly one output model
\verb|sc| which is also an EMF model.  It could also receive many input or
output models, and the models could belong to different modeling frameworks as
well.

\begin{sloppypar}
  The transformation consists of two transformation rules:
  \verb|place2basic-and-or|, and \verb|transition2hyperedge|.  Every rule has
  exactly one parameter denoting the source element.  The \verb|:from| clause
  defines the metamodel type the rule is applicable for.  The \verb|:to| clause
  specifies which target elements have to be created.  Additionally, arbitrary
  constraints could be defined for a rule using a \verb|:when| clause.
  Thereafter, arbitrary code may follow for initializing the newly created
  elements and calling other rules.
\end{sloppypar}

When a rule gets called and is applicable with respect to its declared
\verb|:from| type and \verb|:when| constraint, it creates the elements declared
in \verb|:to| in the target model, and evaluates its body.  In case there is
just one new element declared in \verb|:to|, it returns just that.  If there
are many new elements, it returns them as a vector in their declaration order.
Furthermore, a traceability mapping is created from the source element to the
rule's return value.  If a rule gets called multiple times for a single
element, the second and all following calls just return the result of the first
invocation.


\begin{listing}[H]
  \begin{clojurecode}
(deftransformation initialize-statechart [[pn :emf] [sc :emf]]
  (^:top place2basic-and-or [p]
         :from 'Place
         :to [o 'OR, b 'Basic]
         (eset! b :name (eget p :name))
         (eset! b :rcontains o)
         (eset! b :rnext (map transition2hyperedge
                              (eget p :pret)))
         (eset! b :next  (map transition2hyperedge
                              (eget p :postt))))
  (transition2hyperedge [t]
         :from 'Transition
         :to [he 'HyperEdge]
         (eset! he :name (eget t :name))))
  \end{clojurecode}
  \label{lst:init}
  \caption{The initialization transformation}
\end{listing}

\begin{sloppypar}
  The \verb|place2basic-and-or| rule creates an \verb|OR o| and a
  \verb|Basic b| for any \verb|Place p| it is called with, it sets the name of
  the \verb|Basic| to the name of the \verb|Place|, and assigns the new
  \verb|OR| as container of \verb|b|.  Lastly, it maps the
  \verb|transition2hyperedge| rule over the pre/post-transitions of \verb|p|
  setting the results to \verb|b|'s \verb|rnext|/\verb|next| references.

  The \verb|^:top| metadata at the rule specifies that this rule is called
  automatically for any \verb|Place| in the source Petri-net model \verb|pn|.
  In contrast, the \verb|transition2hyperedge| rule is only called explicitly
  from \verb|place2basic-and-or|.
\end{sloppypar}

The result of such a transformation is always a map of traceability
information.  The keys of the map are keywords denoting the rules, the values
are maps from source elements to rule results.  So in this concrete case, the
traceability map returned by the transformation has the form:

\begin{clojurecode*}{linenos=false}
{:place2basic-and-or   {<place1> [<or1> <basic1>], ...},
 :transition2hyperedge {<transition1> <hyperedge1>, ...}}
\end{clojurecode*}

The function \verb|init-statechart| depicted in Listing~\ref{lst:init-fn} is a
convenience wrapper for applying the transformation.

\begin{listing}[H]
  \begin{clojurecode*}{firstnumber=15}
(defn init-statechart [pn]
  (let [sc (new-model)
        trace (initialize-statechart pn sc)]
    [sc
     (apply hash-map (mapcat (fn [[p [o b]]] [p o])
                             (:place2basic-and-or trace)))
     (apply hash-map (mapcat (fn [[p [o b]]] [p b])
                             (:place2basic-and-or trace)))
     (:transition2hyperedge trace)]))
  \end{clojurecode*}
  \label{lst:init-fn}
  \caption{An initialization transformation wrapper function}
\end{listing}

It receives a Petri-net model \verb|pn|, creates a new empty model \verb|sc|
for the statechart, calls the \verb|initialize-statechart| transformation, and
then mangles the traceability map in order to return a vector of four
components: the initialized statechart model, a map from places to
corresponding \verb|OR| elements, a map from places to corresponding
\verb|Basic| elements, and a map from transitions to corresponding hyperedges.
The first map is required by the reduction transformation, whereas the other
two maps are only used by the unit-tests for the initialization transformation
(see Section~\ref{sec:validation}).


\section{The Reduction Transformation}
\label{sec:reduction-transformation}

The reduction transformation is implemented algorithmically based on FunnyQT's
querying and model manipulation APIs.  It consists of four rules (functions):
\begin{compactenum}
\item The AND-rule as discussed in the case description \cite{pn2sccasedesc},
\item the OR-rule as discussed in the case description,
\item an additional, extension rule assigning hyperedges to the nearest
  \verb|Compound| state containing all their predecessor and successor
  \verb|Basic| states,
\item and a rule creating a \verb|Statechart| with an \verb|AND| top-state if
  the reduction could be completed successfully.
\end{compactenum}


\subsection{The Reduction Function}
\label{sec:reduction-function}

The reduction transformation (a plain function) is shown in
Listing~\ref{lst:reduction-fn}.

It receives a Petri-net model \verb|pn| and calls the statechart initialization
function from Listing~\ref{lst:init-fn}.  As discussed above, this function
returns a vector containing the new statechart and three traceability maps.  By
imitating the result structure in the \verb|let|, the new statechart is
assigned to \verb|sc|, and the map from places to OR objects is assigned to
\verb|place2or|.  The other two maps are not needed for the reduction, so they
are assigned to a variable \verb|_|, which is conventionally used as ``don't
care''.  This technique of binding parts of the contents collections directly
by imitating their structure is known as \emph{destructuring} in the
Lisp-world.

All Clojure datastructures are immutable.  However, the transformation rules
will need to modify the \verb|place2or| traceability map.  Therefore, it is
wrapped in and atom.  An atom is a mutable reference that can be swapped to a
new value atomically.

\begin{listing}[H]
  \begin{clojurecode*}{firstnumber=24}
(defn create-statechart [pn]
  (let [[sc place2or _ _] (init/init-statechart pn)
        place2or (atom place2or)]
    (iteratively (fn []
                   (let [r     (and-rule pn sc prep  place2or)
                         r (or (and-rule pn sc postp place2or) r)
                         r (or (or-rule  pn sc place2or)       r)]
                     r)))
    (create-top sc)
    (assign-hyperedges sc)
    sc))
  \end{clojurecode*}
  \label{lst:reduction-fn}
  \caption{The reduction function implementing the transformation}
\end{listing}


Starting with line 27, the rules are applied.  The higher-order function
\verb|iteratively| takes a function and applies it as long as it returns
logically true\footnote{In Clojure, everything is logically true except for
  \textsf{false} and \textsf{nil}.}

The anonymous function it is called with applies the \verb|and-rule| once for
pre-places and once for post-places, and finally it calls the \verb|or-rule|.
The results of the rules are combined using disjunction in such a way that all
rules get applied in that sequence and the final result \verb|r| is logical
true if at least one rule could be applied.

Lastly, the \verb|create-top| rule creating a \verb|Statechart| element and its
top-AND-State, and the \verb|assign-hyperedges| rule are applied.  The final
result is the new statechart \verb|sc|.


\subsection{Reduction Helper Functions}
\label{sec:reduct-help-functions}

Before discussing the rules, the helper functions depicted in
Listing~\ref{lst:reduction-helpers} are explained.

\begin{listing}[H]
  \begin{clojurecode*}{firstnumber=35}
(defn refs-as-set [ref elem]
  (set (eget-raw elem ref)))

(def postt (partial refs-as-set :postt))
(def pret  (partial refs-as-set :pret))
(def postp (partial refs-as-set :postp))
(def prep  (partial refs-as-set :prep))
  \end{clojurecode*}
  \label{lst:reduction-helpers}
  \caption{Helper functions for the reduction rules}
\end{listing}

The function \verb|refs-as-set| gets a (multi-valued) reference name as a
keyword and some model element, and it returns the value of this reference
coerced to a set\footnote{\textsf{eget-raw} is similar to \textsf{eget}, except
  that the former just returns the EMF collection, i.e., a mutable
  \textsf{EList}, wheras the latter also coerces to an immutable Clojure
  collections.  Because we are coercing to an immutable Clojure set anyway,
  this is unneeded effort here.}.

Then, there are four partial applications of this function where its first
parameter is already preset, thus leaving functions that just receive an
element.  That is, \verb|postt| and \verb|pret| are functions for getting the
set of post- and pre-transitions of a given \verb|Place|, and \verb|postp| and
\verb|prep| are functions for getting the set of post- and pre-places of a
given \verb|Transition|.

\subsection{The AND Rule}
\label{sec:and-rule}

The AND rule is depicted in Listing~\ref{lst:and-rule}.  In contrast to the
Figure~2 in the case description \cite{pn2sccasedesc}, it doesn't delete all
places $q_1$ to $q_n$ to create a new place $p$, but instead it reuses $q_1$ as
$p$ and deletes only $q_2$ to $q_n$.  This is consistent with Louis Rose's EOL
solution, and it even feels more natural at least for algorithmic solutions.

The rule function receives the Petri-net model \verb|pn|, the statechart model
\verb|sc|, either the function \verb|prep| or \verb|postp| as
\verb|prep-or-postp|, and the traceability map atom \verb|place2or|.

\begin{listing}[H]
  \begin{clojurecode*}{firstnumber=42}
(defn and-rule [pn sc prep-or-postp place2or]
  (loop [ts (eallobjects pn 'Transition), applied false]
    (if (seq ts)
      (let [t (first ts), preps-or-postps (prep-or-postp t)]
        (if (> (count preps-or-postps) 1)
          (let [p (first preps-or-postps), prets (pret p), postts (postt p)]
            (if (forall? #(and (= prets  (pret %))
                               (= postts (postt %)))
                         (rest preps-or-postps))
              (let [new-or  (ecreate! sc 'OR), new-and (ecreate! sc 'AND)]
                (eset! new-and :contains (mapv @place2or preps-or-postps))
                (eadd! new-or  :contains new-and)
                (swap! place2or assoc p new-or)
                (doseq [op (rest preps-or-postps)]
                  (edelete! op))
                (recur (rest ts) true))
              (recur (rest ts) applied)))
          (recur (rest ts) applied)))
      applied)))
  \end{clojurecode*}
  \label{lst:and-rule}
  \caption{The AND rule}
\end{listing}

It uses \verb|loop| and \verb|recur| which implement a local tail-recursion,
i.e., a recursion that doesn't consume space on the call-stack.  \verb|loop|
defines the initial bindings, and \verb|recur| restarts the loop with new
bindings.  So initially, \verb|ts| is bound to the sequence of all transitions
in the model, and \verb|applied|, which is used to indicate to the caller if at
least one match has been found, is \verb|false|.

If there are no transitions left\footnote{\textsf{(seq coll) is the canonical
    non-emptiness check in Clojure.}} (the else-branch of the \verb|if| in line
44), the function returns with the current value of \verb|applied|.  If there
are still transitions, the first one is bound to \verb|t| and its set of pre-
or post-places is bound to \verb|preps-or-posts| (line 45).

If there aren't more than one pre- or post-places of \verb|t| (the else branch
of the \verb|if| in line 46), the \verb|loop| gets restarted with the rest of
transitions and the current value of \verb|applied|.  If there are more than
one pre- or post-places, the first one is bound to \verb|p|, and \verb|prets|
and \verb|postts| are its pre- and post-transitions (line 47).

If the other pre- or post-places don't have the same sets of pre- and
post-transitions (the else-branch of the \verb|if| in line 48), the \verb|loop|
is restarted with the remaining transitions and the current value of
\verb|applied|.  However, if the pre- and post-transitition sets are all equal,
the rule matches.  In that case, lines 51 to 56 create a new \verb|OR| and a
new \verb|AND| where the \verb|OR| contains the \verb|AND|, and the \verb|AND|
contains all the \verb|OR| states corresponding to the pre- or post-places of
the current transition \verb|t|\footnote{\textsf{@some-atom atomically
    dereferences an atom resulting in its current value.}}.  The first place
\verb|p| is preserved and its traceability mapping is updated to point to the
new \verb|OR| (line 54).  The other places are deleted in lines 55 and 56.
Finally, the \verb|loop| is restarted with the remaining transitions and an
\verb|applied| value of \verb|true|.

\subsection{The OR Rule}
\label{sec:or-rule}

The OR rule is depicted in Listing~\ref{lst:or-rule}.  In contrast to the case
description, it doesn't delete the places (or corresponding OR states) $q$ and
$r$ to create a new place (or corresponding OR state) $p$, but instead it
reuses $q$ as $p$ and only deletes $r$.

The \verb|or-rule| gets the Petri-net model \verb|pn|, the statechart model
\verb|sc|, and the traceability map atom \verb|place2or|.  It's mechanics for
searching for matches in terms of \verb|loop| and \verb|recur| are almost
identical to the \verb|and-rule|, so the rule is described a bit more concisely
here.  One minor difference is that the variable \verb|ts| is initially bound
to a vector of all transitions.  \verb|eallobjects| returns a lazy sequence,
that is, a sequence where elements are computed (\emph{realized}) when they are
consumed.  Since this rule deletes transitions, the fail-fast EMF model
iterator underlying the lazy sequence will break.  The explicit conversion to a
vector forces that all transitions are computed beforehand.

\begin{listing}[H]
  \begin{clojurecode*}{firstnumber=61}
(defn or-rule [pn sc place2or]
  (loop [ts (vec (eallobjects pn 'Transition)), applied false]
    (if (seq ts)
      (let [t (first ts), preps (prep t), postps (postp t)]
        (if (= 1 (count preps) (count postps))
          (let [q (first preps), r (first postps)]
            (if (and (not (member? r (adjs q :pret :postp)))
                     (not (member? r (adjs q :postt :prep))))
              (let [merger (@place2or q), mergee (@place2or r)]
                (edelete! t)
                (eaddall! q :pret  (eget-raw r :pret))
                (eaddall! q :postt (eget-raw r :postt))
                (edelete! r)
                (eaddall! merger :contains (eget-raw mergee :contains))
                (edelete! mergee)
                (recur (rest ts) true))
              (recur (rest ts) applied)))
          (recur (rest ts) applied)))
      applied)))
  \end{clojurecode*}
  \label{lst:or-rule}
  \caption{The OR rule}
\end{listing}

Lines 63 to 68 specify the application condition of the rule.  If the pre- and
post-place sets of the current transition \verb|t| both contain only a single
place \verb|q| and \verb|r|, respectively, and if \verb|r| is neither reachable
from \verb|q| by traversing the \verb|pret| reference followed by the
\verb|postp| reference, nor reachable from \verb|q| by traversing the
\verb|postt| followed by the \verb|prep| reference, then \verb|t| is a matching
transition.  In that case, the transition \verb|t| is deleted.  \verb|r|'s
\verb|pret| and \verb|postt| references are merged into \verb|q|, and \verb|r|
is deleted.  Similarly, the OR states corresponding to \verb|q| (\verb|merger|)
and \verb|r| (\verb|mergee|) are merged, i.e., the contents of \verb|mergee|
are transferred to \verb|merger|, and \verb|mergee| is deleted.  Finally, the
loop is restarted with the remaining transitions and an \verb|applied| value of
\verb|true|.



\subsection{Extension: The HyperEdge Assignment Rule}
\label{sec:hyperedge-rule}

The hyperedge assignment rule is shown in Listing~\ref{lst:hyperedge-rule}.  It
assigns each \verb|HyperEdge| in the target statechart model to the
\verb|Compound| state that contains all its \verb|rnext| and \verb|next|
states.

The rule receives the statechart model \verb|sc|.

\begin{listing}[H]
  \begin{clojurecode*}{firstnumber=80}
(defn assign-hyperedges [sc]
  (doseq [e (eallobjects sc 'HyperEdge)]
    (eset! e :rcontains
           (first (apply clojure.set/intersection
                         (map #(reachables % [p-+ --<>])
                              (concat (eget e :next) (eget e :rnext))))))))
  \end{clojurecode*}
  \label{lst:hyperedge-rule}
  \caption{The hyperedge assignment rule}
\end{listing}

It iterates over all \verb|HypedEdge| elements in the model, and for each
hyperedge \verb|e|, it sets its \verb|rcontains| reference.  The container
element is determined as follows.

For every \verb|Basic| state in \verb|e|'s \verb|next| and \verb|rnext|
references, the ordered set of containers is computed using a \emph{regular
  path expression}.  The \verb|reachables| function gets the start element of
the search, and a vector describing the path expression.  It returns the
ordered set of elements reachable by a path matching the regular path
expression.  Here, the \verb|--<>| defines that elements should be traversed
towards their container, and by wrapping it in \verb|[p-+ ...]| this iteration
may take place one or many times (transitive closure).  Thus, the ordered
result set contains all containers in the order from nearest to farthest in the
containment hierarchy.  The intersection of all those ordered sets is an
ordered set of all \verb|Compound| states containing all predecessor and
successor \verb|Basic| states of \verb|e|, and its first element is the deepest
one in the overall containment hierarchy.

If the reduction rules didn't terminate with exactly one \verb|Place| left and
one single top-level \verb|OR| state, as it happens with some of the test
models, the intersection above is empty.  In that case, the \verb|rcontains|
reference is set to \verb|nil|, i.e., it is left unset.

\subsection{The Statechart Creation Rule}
\label{sec:statechart-rule}

The final rule of the transformation is the \verb|create-top| rule shown in
Listing~\ref{lst:statechart-rule}.  It receives the target statechart model
\verb|sc| as argument.

\begin{listing}[H]
  \begin{clojurecode*}{firstnumber=86}
(defn create-top [sc]
  (let [top-ors (filter #(not (eget % :rcontains)) (eallobjects sc 'OR))]
    (when (= 1 (count top-ors))
      (let [statechart (ecreate! sc 'Statechart), top (ecreate! sc 'AND)]
        (eset! statechart :topState top)
        (eset! top :rcontains top-ors)))))
  \end{clojurecode*}
  \label{lst:statechart-rule}
  \caption{The statechart creation rule}
\end{listing}

If there is exactly one \verb|OR| state that's not contained in some other
\verb|Compound| state, the reduction process has been applied successfully.  In
that case, a new \verb|Statechart| element \verb|statechart| and a new
\verb|AND| state \verb|top| are created.  \verb|top| is set as \verb|topState|
of the \verb|statechart|, and the contents of \verb|top| are set to the single
top-level \verb|OR| state.


\section{Extension: Validation}
\label{sec:validation}

As an extension to the case (in addition to the hyperedge assignment rule
discussed in Section~\ref{sec:hyperedge-rule}), a small testing project for
validating the result statechart models of this transformation case has been
implemented.  It is published at
github\footnote{\url{https://github.com/tsdh/ttc-2013-pn2sc-validation}} where
also its usage documentation can be found.

This project is capable of testing the correctness of the result statecharts of
provided eleven Petri-nets, and it also checks the result statecharts generated
from the 15 performance testing Petri-nets.

The following constraints are checked for the primary test cases:

\begin{compactenum}
\item For the test cases where a complete reduction is feasible, there has to
  be exactly one \verb|Statechart| element containing exactly one \verb|AND|
  state containing the top-most \verb|OR| state created by the reduction rules.
\item For every element type in the statechart metamodel, the expected number
  of instances is checked against the actual number of instances.
\item The expected containment hierarchy is checked against the actual
  containment hierarchy.  If at least one hyperedge is contained in some
  \verb|Compound| state, the correct containments of hyperedges is also tested.
  Else, a warning is issued referring to the hyperedge containment extension.
\item The expected contents of the \verb|rnext| and \verb|next| references of
  each \verb|HyperEdge| are checked against their actual contents.
\end{compactenum}

For the large results of the performance test cases, only the checks 1 and 2
are performed.


\section{Running the Transformation on SHARE}
\label{sec:run-transformation}

The FunnyQT solution of this case\footref{fn:github} (and the other cases) are
installed on the SHARE image \verb|Ubuntu12LTS_TTC13::FunnyQT.vdi|.  Running
the solution is simple.

\begin{compactenum}
\item Open a terminal.
\item Change to the Petri-Nets to Statecharts project:

  \verb|$ cd ~/Desktop/FunnyQT_Solutions/ttc-2013-pn2sc/|
\item Run the test cases:

  \verb|$ lein test|
\end{compactenum}

This will run the complete transformation (initialization + reduction) on all
provided test Petri-net models (the 11 main test cases and the performance test
cases) and print the execution times.  The result models are also validated
using the testing project discussed in Section~\ref{sec:validation}.  The
result models and visualizations of the main test cases' results are saved to
the \verb|results| directory.

Furthermore, the stand-alone initialization transformation is applied to every
provided model as well.  Again, the times needed to apply the transformation
are printed.


\section{Evaluation}
\label{sec:evaluation}

In this section, the solution is evaluated according to the evaluation criteria
listed in the case description \cite{pn2sccasedesc}.

\paragraph{Transformation correctness.}

The validation project discussed in Section~\ref{sec:validation} that has been
implemented as an extension to this case allows for testing the result
statechart models.  For the main test cases, every important aspect of the
result models including the containment hierarchy and the predecessors and
successors of hyperedges are checked, and for the performance test cases, only
the number of instances of every metamodel class is checked.  All tests pass
for the result models of this solution.  Similarly, all tests pass for the
result models created by the reference GrGen.NET solution.

The validation project has also been tested with intentionally slightly wrong
models, e.g., some \verb|next| link is missing at some hyperedge, there's some
additional element, or an element is contained by the wrong \verb|Compound|
state.  In all those cases, an assertion of the validation project failed.  So
there's a high confidence that if the result models pass the tests, the
transformation producing them is correct.

\paragraph{Transformation performance.}


Table~\ref{tab:eval-times} shows the evaluation times of the initialization
tranformation, the reduction transformation, and the complete transformation
involving initialization and reduction.  The times needed for loading and
saving the models from/to XMI files, the times needed for validation, and the
times needed for creating visualizations are excluded.

The measurements have been done directly on the SHARE demo.  As stated in
Section~\ref{sec:run-transformation}, \verb|lein test| will first evaluate the
complete transformation on all models, and then it'll evaluate the
initialization transformation again on all models.  The evaluation times
printed there are contained in the tables third and first column, respectively.
The second column is the difference between the value in the third column and
the value in the first column.


\begin{table}[H]
  \centering
  \begin{tabular}{| c | r | r |r |}
    \hline
    \textbf{Test Case} & \textbf{Init. only} & \textbf{Red. only} & \textbf{Init. \& Red.}\\
    \hline
    1        & 1 ms    & 81 ms & 82 ms\\
    2        & 1 ms    & 37 ms & 38 ms\\
    3        & <1 ms   & 23 ms & 24 ms\\
    4        & <1 ms   & 9 ms & 10 ms\\
    5        & <1 ms   & 2 ms  & 3 ms\\
    6        & <1 ms   & 2 ms  & 3 ms\\
    7        & <1 ms   & 15 ms & 16 ms\\
    8        & <1 ms   & 20 ms & 21 ms\\
    9        & <1 ms   & 7 ms  & 8 ms\\
    10       & <1 ms   & 7 ms  & 8 ms\\
    11       & <1 ms   & 9 ms  & 10 ms\\
    \hline
    sp200    & 12 ms   & 140 ms    & 152 ms\\
    sp300    & 18 ms   & 92 ms     & 110 ms\\
    sp400    & 24 ms   & 97 ms     & 121 ms\\
    sp500    & 28 ms   & 107 ms    & 125 ms\\
    sp1000   & 69 ms   & 223 ms    & 292 ms\\
    sp2000   & 109 ms  & 267 ms    & 376 ms\\
    sp3000   & 153 ms  & 448 ms    & 601 ms\\
    sp4000   & 226 ms  & 517 ms    & 743 ms\\
    sp5000   & 274 ms  & 743 ms    & 1017 ms\\
    sp10000  & 553 ms  & 1509 ms   & 2062 ms\\
    sp20000  & 960 ms  & 3652 ms   & 4612 ms\\
    sp40000  & 1887 ms & 9457 ms   & 11344 ms\\
    sp80000  & 3849 ms & 26442 ms  & 30291 ms\\
    sp100000 & 4755 ms & 35639 ms  & 40394 ms\\
    sp200000 & 9407 ms & 105257 ms & 114664 ms\\
    \hline
  \end{tabular}
  \caption{Evaluation times on SHARE}
  \label{tab:eval-times}
\end{table}

That the first test cases evaluate a bit slower than subsequent ones is
probably caused by the JVM just-in-time compiling critical code paths.
Fluctuations in the evaluation times of models of comparable size are also
partly caused by JVM-internals such as garbage collection.  Nevertheless, the
order of magnitude is stable across multiple runs of the transformations.

Anyhow, the evaluation times are quite good, both absolute as well as relative
compared to the input model sizes.  The initialization transformation scales
linearly with the size of the input Petri-net models.  The reduction
transformation also scales in that order of magnitude.  When taking the time
needed for the \verb|sp1000| model as a baseline, then the \verb|sp10000| model
is transformed in $0.7n$ milliseconds for $n$ being the time expected when
assuming a linear correlation between model size and execution time.  The
\verb|sp40000| model takes $n$ milliseconds, the \verb|sp100000| takes $1.6n$
milliseconds, and the \verb|sp200000| model takes $2.4n$ milliseconds.

\paragraph{Transformation understandability.}

Although the solution requires some understanding of Clojure, it shouldn't be
hard to get a grasp on it.

The initialization transformation uses a FunnyQT facility allowing to specify
typical model transformations with a syntax and semantics similar to ATL or
ETL, so people knowing these languages should feel right at home.

The reduction transformation is a bit more complex, but the application
conditions of the rules and the actions that are performed are taken quite
literally from the case description with the exception that some elements are
preserved and merged instead of replaced.

One important aspect with respect to understandability is also the fact that
the transformations are very concise.  In total, the initialization and the
reduction transformation need just the 91 lines of code that was depicted
completely in the previous sections.  The initialization transformation
including its wrapper function mangling the traceability mappings is just 23
lines of code.  The complete reduction transformation with its four rules, the
few helper functions, and the function iteratively applying the rules as long
as possible amounts to 68 lines of code.


\paragraph{Bonus criteria.}

The bonus tasks dealing with \emph{verification} and \emph{simulation support}
haven't been tackled.

The initialization transformation could be extended quite easily to deal with
the \emph{change propagation} scenarios, simply because FunnyQT transformations
are written in the full-fledged programming language Clojure having access to
any JVM library's features.  Thus, adapters performing the required changes on
the target statechart could be registered to handle notifications about new,
deleted, or updated elements in the source Petri-net using the standart EMF
notification framework.  Since the initialization transformation created a
complete traceability mapping, referring to previously created elements is
easy.

There is no support special support for \emph{reversing the transformation}, or
for defining the transformation bidirectionally in the first place.

Proper \emph{debugging support} is also not yet ready for prime-time in the
Clojure world.  There are some attempts at debuggers allowing to set
breakpoints and examine the lexical extent around the breakpoint, but those are
not too usable right now.  Another difficulty with functional languages
involving some kind of laziness is that errors might be signaled at a location
very different to where the bug is actually manifested in the source code.
Nevertheless, FunnyQT has rather good model visualization tools that have been
used while programming the reduction rules in order to visualize the matching
elements when a rule has been applicable.



\bibliographystyle{alpha}
\bibliography{ttc13-funnyqt-pn2sc}


\end{document}




%%% Local Variables:
%%% mode: latex
%%% TeX-engine: pdflatex-shell-escape
%%% TeX-master: t
%%% End:
